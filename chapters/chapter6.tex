\chapter{Chinese Scholarship --- Part I}
Not long ago a body of missionaries created a great deal of amusement by styling themselves, on the cover of some scientific tracts, as ``famous savants" \emph{su ju} (宿儒).
The idea was of course extremely ridiculous.
There is certainly not one Chinaman in the whole Empire who would venture to arrogate to himself the Chinese word \emph{ju}, which includes in it all the highest attributes of a scholar or literary man.
We often hear, however, a European spoken of as a Chinese scholar.
In the advertisement of the \emph{China Review}, we are told that ``among the missionaries a high degree of Chinese scholarship is assiduously cultivated."
A list is then given of regular contributors, ``all," we are to believe, ``well-known names, indicative of sound scholarship and thorough mastery of their subject."

Now in order to estimate the high degree of scholarship said to be assiduously cultivated by the missionary bodies in China, it is not necessary to take such high ideal standards as those propounded by the German Fichte in his lectures upon the Literary Man, or the American Emerson in his Literary Ethics.
The late American Minister to Germany, Mr. Taylor, was acknowledged to be a great German scholar; but though an Englishman who has read a few plays of Schiller, or sent to a magazine some verses translated from Heine, might be thought a German scholar among his tea drinking circles, he would scarcely have his name appear as such in print or placard.
Yet among Europeans in China the publication of a few dialogues in some provincial \emph{patois}, or collection of a hundred proverbs, at once entitles a man to be called a Chinese scholar.
There is, of course, no harm in a name, and,\footnote{. in the original book} with the exterritorial clause in the treaty, an Englishman in China might with impunity call himself Confucius if so it pleases him.

We have been led to consider this question because it is thought by some that Chinese scholarship has passed, or is passing, the early pioneering, and is about to enter a new stage, when students of Chinese will not be content with dictionary-compiling or such other brick-carrying work, but attempts will be made at works of construction, at translations of the most perfect specimens of the national literature, and not only judgment, but final judgment, supported with reasons and arguments, be passed upon the most venerated names of the Chinese literary Pantheon.
We now propose to examine: 1st, how far it is true that the knowledge of Chinese among Europeans is undergoing this change; 
2ndly, what has already been done in Chinese scholarship; 
3rdly, what is the actual state of Chinese scholarship at the present day; and in the last place, to point out what we conceive Chinese scholarship should be.
It has been said that a dwarf standing upon the shoulders of a giant is apt to imagine himself of greater dimensions than the giant; still, it must be admitted that the dwarf, with the advantage of his position, will certainly command a wider and more extensive view.
We will, therefore, standing upon the shoulders of those who have preceded us, take a survey of the past, present, and future of Chinese scholarship; and if, in our attempt, we should be led to express opinions not wholly of approval of those who have gone before us, these opinions, we hope, may not be construed to imply that we in any way plume ourselves upon our superiority: we claim only the advantage of our position.

First, then, that the knowledge of Chinese among Europeans has changed, is only so far true, it seems to us, that the greater part of the difficulty of acquiring a knowledge of the language has been removed.
``The once prevalent belief," says Mr. Giles, ``in the great difficulty of acquiring a colloquial knowledge, even of a single Chinese dialect has long since taken its place among other historical fictions.''
Indeed, even with regard to the written language, a student in the British Consular Service, after two years' residence in Peking and a year or two at a Consulate, can now readily make out at sight the general meaning of an ordinary despatch\footnote{dispatch in the original book}.
That the knowledge of Chinese among foreigners in China has so far changed, we readily admit; but what is contended for beyond this we feel very much inclined to doubt.

After the early Jesuit missionaries, the publication of Dr. Morrison's famous dictionary is justly regarded as the \emph{point de d\^epart}\footnote{point de depart in the original book} of all that has been accomplished in Chinese scholarship.
The work will certainly remain a standing monument of the earnestness, zeal and conscientiousness of the early Protestant Missionaries.
After Morrison came a class of scholars of whom Sir John Davis and Dr. Gutzlaff might be taken as representatives.
Sir John Davis really knew no Chinese, and he was honest enough to confess it himself.
He certainly spoke Mandarin and could perhaps without much difficulty read a novel written in that dialect.
But such knowledge as he then possessed, would now-a-days scarcely qualify a man for an interpretership in any of the Consulates.
It is nevertheless very remarkable that the notions about the Chinese of most Englishmen, even to this day, will be found to have been acquired from Sir John Davis's book on China.
Dr. Gutzlaff perhaps knew a little more Chinese than Sir John Davis; but he attempted to pass himself off as knowing a great deal more than he did.
The late Mr. Thomas Meadows afterwards did good service in exposing the pretension of Dr. Gutzlaff, and such other men as the missionaries Huc\footnote{Hue in the original book} and Du Halde. 
After this, it is curious to find Mr. Boulger, in his recent History of China, quoting these men as authorities.

In France, R\'emusat was the first to occupy a Chair of Chinese Professorship in any European University.
Of his labours we are not in a position to express an opinion.
But one book of his attracted notice: it was a translation of a novel, ``The Two Cousins."
The book was read by Leigh Hunt, and by him recommended to Carlyle, and by Carlyle to John Stirling, who read it with delight, and said that the book was certainly written by a man of genius, but ``a man of genius after the dragon pattern."
The\footnote{the in the original book} \emph{Ju Kiao Li}\marginpar{Ju Kiao Li: 玉娇梨}, as the novel is called in Chinese, is a pleasant enough book to read, but it takes no high\footnote{High in the original book} place even among the inferior class of books of which it is a specimen.
Nevertheless it is always pleasant to think that thoughts and images from the brain of a Chinaman have actually passed through such minds as those of Carlyle and Leigh Hunt.

After R\'emusat followed Stanislas Julien and Pauthier.
The German poet Heine says that Julien made the wonderful and important discovery that Mons. Pauthier did not understand Chinese at all and the latter, on the other hand, also made a discovery, namely that Monsieur Julien knew no Sanscrit.
Nevertheless the pioneering work done by these writers was very considerable.
One advantage they possessed was that they were thorough masters of their own language.
Another French writer might be mentioned, Mons. D'Harvey St. Denys,\footnote{' in the original book} whose translation of the T'ang poets is a breach made into one department of Chinese literature in which nothing has been done before or since.

In Germany Dr. Plath of Munich published a book on China, which he entitled ``Die Manchurei."
Like all books written in Germany, it is a solid piece of work thoroughly well done.
Its evident design was to give a history of the origin of the present Manchu dynasty in China.
But the latter portions of the book contain information on questions connected with China, which we know not where to find in any other book written in a European language.
Such work as Dr. Williams's Middle Kingdom' is a mere nursery story-book compared with it.
Another German Chinese scholar is Herr von Strauss, formerly the Minister of a little German principality which has since 1866 been swallowed up by Prussia.
The old Minister in his retirement amused himself with the study of Chinese.
He published a translation of Lao Tzu, and recently of the Shih King.
Mr. Faber, of Canton, speaks of some portions of his Lao Tzu as being perfect.
His translation of the Odes is also said to be very spirited.
We have, unfortunately, not been able to procure these books.

The scholars we have named above may be regarded as sinologues\footnote{Sinologues in the original book} of the earliest period, beginning with the publication of Dr. Morrisons's dictionary.
The second period began with the appearance of two standard works: 1st, the Tzu Erh Chih of Sir Thomas Wade; 2nd, the Chinese Classics of Dr. Legge.

As to the first, those who have now gone beyond the Mandarin colloquial in their knowledge might be inclined to regard it slightingly\footnote{lightingly in the original book}.
But it is, notwithstanding, a great work --- the most perfect, within the limits of what was attempted, of all the English books that have been published on the Chinese language.
The book, moreover, was written in response to a crying necessity of the time.
Some such book had to be written, and lo!\footnote{lo! is missing in the original book} it was done, and done in a way that took away all chance of contemporary as well as future competition.

That the work of translating the Chinese Classics had to be done, was also a necessity of the time, and Dr. Legge has accomplished it, and the result is a dozen huge, ponderous tomes.
The quantity of work done is certainly stupendous, whatever may be thought of the quality.
In presence of these huge volumes we feel almost afraid to speak.
Nevertheless, it must be confessed that the work does not altogether satisfy us.
Mr. Balfour justly remarks that in\footnote{m in the original book} translating these classics a great deal depends upon the terminology employed by the translator.
Now we feel that the \emph{terminology} employed by Dr. Legge is harsh, crude, inadequate, and in some places, almost unidiomatic.
So far for the form.
As to the matter, we will not hazard our own opinion, but will let the Rev. Mr. Faber of Canton speak for us.
``Dr. Legge's own notes on Mencius," he says, "show that Dr. Legge has not a philosophic understanding of his author.''
We are certain that Dr. Legge could not have read and translated these works without having in some way tried to conceive and shape to his own mind the teaching of Confucius and his school as a connected whole; yet it is extraordinary that neither in his notes nor in his dissertations has Dr. Legge let slip a single phrase or sentence to show what he conceived the teaching of Confucius really to be, as a philosophic whole.
Altogether, therefore, Dr. Legge's judgment on the value of these works cannot by any means  be accepted as final, and the translator of the Chinese Classics is yet to come.
Since the appearance of the two works above mentioned, many books have been written on China: a few, it is true, of really great scholastic importance; but none, we believe showing that Chinese scholarship has reached an important turning point.

First, there is Mr. Wylie's ``Notes on Chinese Literature."
It is, however, a mere catalogue, and not a book with any literary pretension at all.
Another is the late Mr. Mayers's\footnote{Mayers' in the original book} ``Chinese Readers Manual."
It is certainly not a work that can lay claim to any degree of perfection.
Nevertheless, it is a very great work, the most honest conscientious and unpretending of all the books that have been written on China.
Its usefulness, moreover, is inferior only to the Tzu-Erh-Chi of Sir Thomas Wade.

Another Chinese scholar of note is Mr. Herbert A. Giles of the British Consular Service.
Like the early French sinologues\footnote{Sinologues in the original book}, Mr. Giles possesses the enviable advantage of a clear, vigorous, and beautiful style.
Every object he touches upon becomes at once clear and luminous.
But with one or two exceptions, he has not  been quite fortunate in the choice of subjects worthy of his pen.
One exception is the ``Strange Stories\footnote{Stones in the original book} from a Chinese Studio,\footnote{. in the original book}" which may be taken as a model of what translation from the Chinese should be.
But the \emph{Liao-ckai-chih-i}\marginpar{Liao-ckai-chih-i: 聊斋志异}, a remarkably beautiful literary work of art though it be, belongs yet not to the highest specimens of Chinese literature.

Next to Dr. Legge's labours, Mr. Balfour's recent translation of the Nan-hua King of Chuang-tzu is a work of certainly the highest ambition.
We confess to having\footnote{have in the original book} experienced, when we first heard the work announced, a degree of expectation and delight which the announcement of an Englishman entering the Hanlin College would scarcely have raised in us.
The Nan-hua King is acknowledged by the Chinese to be one of the most perfect of the highest specimens of their national literature.
Since its appearance two centuries before the Christian era, the influence of the book upon the literature of China is scarcely inferior to the works of Confucius and his schools; while its effect upon the language and spirit of the poetical and imaginative literature of succeeding dynasties is almost as exclusive as that of the Four Books and Five Chinese upon the philosophical works of China.
But Mr. Balfour's work is not a translation at all; it is simply a \emph{mistranslation}.
This, we acknowledge, is a heavy, and for us, daring judgment to pass upon a work upon which Mr. Balfour must have spent many years.
But we have ventured it, and it will be expected of us to make good our judgment.
We believe Mr. Balfour would hardly condesend\marginpar{\scriptsize condesend [sic: condescend?]} to join issue with us if we were to raise the question of the true interpretation of the philosophy of Chuang-tzu.
``But," --- we quote from the Chinese preface of Lin Hsi-chung, a recent editor of the Nan-hua King --- ``in reading a book, it is necessary to understand first the meaning of each single word: then only can you construe the sentences, then only can you perceive the arrangement of the paragraphs; and then, last of all, can you get at the central proposition of the whole chapter."
Now every page of Mr. Balfour's translation bears marks that he has not understood the meaning of many single words, that he has not construed the sentences correctly, and that he has missed the arrangement of the paragraphs.
If these propositions which we have assumed can be proved to be true, as they can easily be done, being merely points regarding rules of grammar and syntax, it then follows very clearly that Mr. Balfour has missed the meaning and central proposition of whole chapters.

But of all the Chinese scholars of the present day we are inclined to place the Reverend Mr. Faber of Canton at the head.
We do not think that Mr. Faber's labours are of more scholastic value or a higher degree of literary merit than the works of others, but we find that almost every sentence he has written shows a grasp of literary and philosophic principles such as we do not find in any other scholar of the present time.
What we conceive these principles to be we must reserve for the next portion of the present paper, when we hope to be able to state the methods, aims, and objects of Chinese scholarship.

