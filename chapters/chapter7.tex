\chapter{Chinese Scholarship --- Part II}
Mr.~Faber has made the remark that the Chinese do not understand any systematic method of scientific enquiry.
Nevertheless in one of Chinese Classics, called ``Higher Education,''\cite{num26} \marginpar{\scriptsize Higher Education: 大学} a work which is considered by most foreign scholars as a Book of Platitudes, a concatenation is given of the order in which the systematic study of a scholar should be pursued.
The student of Chinese cannot perhaps do better than follow the course laid down in that book namely, to begin his study  with the individual, to proceed from the individual to the family, and from the family to the Government. \marginpar{\scriptsize 《礼记\cdot大学》 修身齐家治国平天下}

First, then: it is necessary and indispensable that the student should endeavour to arrive at a just knowledge of the principles of individual conduct of the Chinese.
Secondly, he will examine and see how these principles are applied and carried out in the complex social relations and family life of the people.
Thirdly, he will be able then to give his attention, and direct his study, to the government and administrative institutions of the country.
Such a programme as we have indicated, can, of course, be followed out only in general outline; to carry it fully out would require the devotion and undivided energies of almost a whole lifetime.
But we should certainly refuse to consider a man a Chinese scholar or a\marginpar{\scriptsize a [sic: an]} attribute to him any high degree of scholarship, unless he had in some way made himself familiar with the principles above indicated.
The German poet Goethe says: ``In the works of man, as in those of nature, what is really deserving of attention, above everything, is --- the \emph{intention}."
Now in the study of national character, it is also of the first importance to pay attention, not only to the actions and practice of the people, but also to their notions and theories; to get a knowledge of what they consider as good and what as bad, what they regard as just and what as unjust, what they look upon as beautiful and what as not beautiful, and how they distinguish wisdom from foolishness.
This is what we mean when we say that the student of Chinese should study the principles of individual conduct.
In other words, we mean to say that you must get at the \emph{national ideals}.
If it is asked how this is to be attained: we answer, by the study of the national literature, in which revelations of the best and highest as well as the worst side of the character of a people can be read.
The one object, therefore, which should engage the attention of the foreign student of Chinese, is the standard national literature of the people: whatever preparatory studies it may by necessary for him to go through should serve only as means towards the attainment of that one object.
Let us now see how the student is to study the Chinese literature.

``The civilisations of Europe," says a German writer, ``rest upon those of Greece, Rome and Palestine; the Indians and Persians are of the same Aryan stock \marginpar{\scriptsize Aryan stock: 亚利安血统} as the people of Europe, and are therefore related; and the influence of the intercourse with the Arabs during the Middle Ages, upon European culture has not even to this day, altogether disappeared."
But as for the Chinese, the origin and development of their civilisation rest upon foundations altogether foreign to the culture of the people of Europe.
The foreign student of Chinese literature, therefore, has all the disadvantages to overcome which must result from the want of community of primary ideas and notions.
It will be necessary for him, not only to equip himself with these foreign notions and ideas, but also, first of all, to find their equivalents in the Europe languages, and if these equivalents do not exist, to disintegrate them, and to see to which side of the universal nature of man these ideas and notions may be referred.
Take, for instance, those Chinese words of constant recurrence in the Classics, and generally translated into English as ``benevolence," (仁) ``justice,'' (義) and ``propriety'' (禮).
Now when we come to take these English words together with the context, we feel that they are not adequate: they do not connote all the ideas the Chinese words contain.
Again, the word ``humanity," is perhaps the most exact equivalent for the Chinese word translated ``benevolence;" but then, ``humanity" must be understood in a sense different from its idiomatic use in the English language.
A venturesome\footnote{venture- some in the original book} translator would use the ``love" and ``righteousness" of the Bible, which are perhaps as exact as any other, having regard both for the sense of the words and the idiom of the language.
Now, however, if we disintegrate and refer the primary notions which these words convey, to the universal nature of man, we get, at once, at their full significance: namely, ``the good'', ``the true'', and ``the beautiful."

But, moreover, the literature of a nation, if it is to be studied at all, must be studied systematically and as one connected whole, and not fragmentarily and without plan or order, as it has hitherto been done by most foreign scholars.
``It is,'' says Mr.~Matthew Arnold, \marginpar{\scriptsize Matthew Arnold: (1822-1888) an English poet and culture critic.} ``it is through the apprehension, either of all literature, --- the entire history of the human spirit, --- or of a single great literary work, as a connected whole that the real power of literature makes itself felt."
Now how little, we have seen, do the foreign students conceive the Chinese literature as a whole!
How little, therefore, do they get at its significance!
How little, in fact, do they know it!
How little does it become a power in their hands, towards the understanding of the character of the people!
With the exception of the labours of Dr.~Legge and of one or two other scholars, the people of Europe know of the Chinese literature principally through the translations of novels, and even these not of the best, but of the most commonplace of their class.
Just fancy, if a foreigner were to judge of the English literature from the works of Miss Rhoda Broughton, \marginpar{\scriptsize Rhoda Broughton: (1840-1920) a Welsh novelist and short story writer.} or that class of novels which form the reading stock of school-boys and nursery-maids!
It was this class of Chinese literature which Sir Thomas Wade must have had in his mind, when in his wrath he reproached the Chinese with ``tenuity of intellect."

Another extraordinary judgment which used to be passed upon Chinese literature was, that it was excessively over-moral. 
Thus the Chinese people were actually accused of over morality, while at the same time most foreigners are pretty well agreed that the Chinese are a nation of liars!
But we can now explain this by the fact that, besides the trashy novels we have already noticed, the work of translation among students of Chinese was formerly confined exclusively to the Confucian Classics.
Nevertheless, there are of course a great many other things in these writings besides morality, and, with all deference to Mr.~Balfour, we think that ``the admirable doctrines" these books contain are decidedly not ``utilitarian and wordly\footnote{worldly in the original book}" \marginpar{\scriptsize wordly [sic: worldly]} as they have been judged to be.
We will just submit two sentences and ask Mr.~Balfour if he really thinks them ``utilitarian and mundane."
``He who sins against Heaven," said Confucius in answer to a Minister, ``he who sins against Heaven has no place where he\footnote{He in the original book} can turn to and pray.''
Again, Mencius says: ``I love life, but I also love righteousness: but if I cannot keep them both, I would give up life and choose righteousness."

We have thought it worth while \marginpar{\scriptsize worth while [sic: worthwhile]} to digress so far in order to protest against Mr.~Balfour's judgment, because we think that such smart phrases as ``a bondslave to antiquity," ``a past-master in casuistry" should scarcely be employed in a work purposely philosophical, much less applied to the most venerated name in China.
Mr.~Balfour was probably led astray by his admiration of the Prophet of Nan-hua, and, in his eagerness to emphasise the superiority of the Taoist over the orthodox school, he has been betrayed into the use of expressions which, we are sure, his calmer judgment must condemn.

But to return from our digression. 
We have said that the Chinese literature must be studied as a connected whole.
Moreover we have noted that the people of Europe are accustomed to conceive and form their judgment of the literature of China solely from those writings with which the name of Confucius is associated; but, in fact, the literary activity of the Chinese had only just begun with the labours of Confucius, and has since continued through eighteen dynasties, including more than two thousand years.
At the time of Confucius, the literary form of writing was still very imperfectly understood.

Here let us remark that, in the study of a literature, there is one important point to be attended to, but which has hitherto been completely lost sight of by foreign students of Chinese; namely, the \emph{form} of the literary writings.
``To be sure," said the poet Wordsworth, \marginpar{\scriptsize William Wordsworth: (1770-1850) a major English Romantic poet who helped to launch the Romantic Age in English literature.} ``it was the matter, but then you know the \emph{matter} always comes out of the \emph{manner}."
Now it is true that the early writings with which the name of Confucius is associated do not pretend to any degree of perfection, as far as the literary form is concerned: they are considered as classical or standard works not so much for their classical elegance of style or perfection of literary form, as for the value of the matter they contain.
The father of Su Tung-po, \marginpar{\scriptsize Su Tung-po: 苏东坡} of the Sung dynasty, remarks that something approaching to the formation of a prose style may be traced in the dialogues of Mencius.
Nevertheless Chinese literary writings, both in prose and poetry, have since been developed into many forms and styles.
The writings of the Western Hans, \marginpar{\scriptsize Western Hans: 西汉} for instance, differ from the essays of the Sung period, much in the same way as the prose of Lord Bacon \marginpar{\scriptsize Francis Bacon: (1561-1626) an English philosopher, author.} is different from the prose of Addison \marginpar{\scriptsize Joseph Addison: (1672-1719) an English essayist, poet and playwriter.} or Goldsmith. \marginpar{\scriptsize Oliver Goldsmith: (1728-1774) an Irish novelist, poet and playwriter.}
The wild exaggeration and harsh diction of the poetry of the six dynasties are as unlike the purity, vigour, and brilliancy of the T'ang poets as the early weak and immature manner of Keats \marinpar{\scriptsize John Keats: (1795-1821) an English Romantic poet. 济慈} is unlike the strong, clear\footnote{dear in the original book}, and correct splendour of Tennyson. \marginpar{\scriptsize Alfred Tennyson: (1809-1892) Poet Laureate of Great Britain and Ireland, one of the most popular British poets.}

Having thus, as we have shown, equipped himself with the primary principles and notions of the people, the student will then be in a position to direct his study to the social relations of the people; to see how\footnote{bow in the original book} these principles are applied and carried out. 
But the social institutions, manners and customs of a people do not grow up, like mushrooms, in a night, but are developed and formed into what they are, through long centuries.
It is therefore necessary to study the history of the people.
Now the history of the Chinese people is as yet almost unknown to European scholars.
The so-called History of China, by Mr.~Demetrius Boulger, published recently, is perhaps the worst history that could have been written of a civilised people like the Chinese.
Such a history as Mr.~Boulger has written might be tolerated if written of some such savage people as the Hottentots. \marginpar{\scriptsize Hottentots: Khoikhoi. A racial term.}
The very fact that such a history of China could have been published, serves only to show how very far from being perfect yet is the knowledge of Chinese among Europeans.
Without a knowledge of their history, therefore, no correct judgment can be formed of the social institutions of a people.
Such works as Dr.~Williams's Middle Kingdom and other works on China from want of such knowledge, are not only useless for the purpose of the scholar, but are even misleading for the mass of general readers.
Just to take one instance, --- the social ceremony of the people.
The Chinese are certainly a ceremonious people, and it is true that they owe this to the influence of the teaching of Confucius.
Now Mr.~Balfour may speak of the pettifogging observances of a ceremonial life as much as he pleases; nevertheless, even ``the bows and scrapes \marginpar{\scriptsize bow and scrape: 鞠躬作揖} of external decorum," as Mr.~Giles calls them, have their roots deep in the universal nature of man, in that side of human nature, namely, which we have defined as the sense of the beautiful.
``In the use of ceremony,'' says a disciple of Confucius, ``what is important, is to be natural; this is what is really beautiful in the ways of the ancient Emperors."
Again, it is said somewhere in the Classics: ``Ceremony is simply the expression of reverence." (the \emph{Ehrfurcht}\footnote{Ehrf\"urcht in the original book} of Goethe's \emph{Wilhelm\footnote{Wilk elm in the original book} Meister}.)
We now see how evident it is that a judgment of the manners and customs of a nation should be founded upon the knowledge of the moral principles of the people.
Moreover the study, of the Government and political institutions of a country, --- which, we have said should be reserved by the student to the last stage of his labours, --- must also be founded upon an understanding of their philosophical principles and a knowledge of their history.

We will conclude with a quotation from ``The Higher Education," or the Book of Platitudes, as foreigners consider it.
``The Government of the Empire,'' it is said in that book, ``should begin with the proper administration of the State; the administration of the State begins with the regulation of the family; the regulation of the family begins with the cultivation of the individual.''
This, then, is what we mean by Chinese Scholarship.

{\scriptsize This article on Chinese Scholarship was written and published in the ``N. C. Daily News" in Shanghai in 1884. --- Exactly thirty years ago!}
