\chapter{The Chinese Language}
All foreigners who have tried to learn Chinese say that Chinese is a very difficult language.
But is Chinese a difficult language?
Before, however, we answer this question, let us understand what we mean by the Chinese language.
There are, as everybody knows, two languages --- I do not mean dialects, --- in China, the spoken and the written language.
Now, by the way, does anybody know the reason why the Chinese insist upon having these two distinct, spoken and written languages?
I will here give you the reason.
In China, as it was at one time in Europe when Latin was the learned or written language, the people are properly divided into two distinct classes, the educated and the uneducated.
The colloquial or spoken language is the language for the use of the uneducated, and the written language is the language for the use of the really educated.
In this way \emph{half educated} people do not exist in this country.
That is the reason, I\footnote{1 in the original book} say, why the Chinese insist upon having two languages.
Now think of the consequences of having half educated people in a country.
Look at Europe and America today.
In Europe and America since, from the disuse of Latin, the sharp distinction between the spoken and the written language has disappeared, there has arisen a class of half educated people who are allowed to use the same language as the really educated people, who talk of civilisation, liberty, neutrality, militarism and panslavinism without in the least understanding what these words really mean.
People say that Prussian Militarism is a danger to civilisation.
But to me it seems, the half educated man, the mob of half educated men in the world today, is the real danger to civilisation.
But that is neither here nor there.

Now to come to the question: is Chinese a difficult language?
My answer is, yes and no.
Let us first take the spoken language.
The Chinese spoken language, I say, is not only \emph{not} difficult, but as compared\footnote{com. pared in the original book} with the half dozen languages that I know, --- the easiest language in the world except, --- Malay.
Spoken Chinese is easy because it is an extremely simple language.
It is a language without case, without\footnote{with- out in the original book} tense, without regular and irregular verbs; in fact without grammar, or any rule whatever.
But people have said to me that Chinese is difficult even because of its simplicity; even because it has no rule or grammar.
That, however, cannot be true.
Malay like Chinese, is also a simple language without grammar or rules; and yet Europeans who learn it, do not find it difficult.
Thus in itself and for the Chinese colloquial or spoken Chinese at least is not a difficult language.
But for educated Europeans and especially for half educated Europeans who come to China, even colloquial or spoken Chinese is a very difficult language: and why?
Because spoken or colloquial Chinese is, as I\footnote{1 in the original book} said, the language of uneducated men, of thoroughly uneducated men; in fact the language of a child.
Now as a proof of this, we all know how easily European children learn colloquial or spoken Chinese, while learned philogues and sinologues insist in saying that Chinese is so difficult.
Chinese, colloquial Chinese, I say again is the language of a child.
My first advice therefore to my foreign friends who want to learn Chinese is ``Be ye like little children, you will then not only enter into the Kingdom of Heaven, but you will also be able to learn Chinese.''

We now come to the written or book language, written Chinese.
But here before I go further, let me say there are also different kinds of written Chinese.
The Missionaries class these under two categories and call them easy \emph{wen li} and difficult \emph{wen li}.
But that, in my opinion, is not a satisfactory classification.
The proper classification, I think, should be, plain dress written Chinese; official uniform Chinese;  and full court dress Chinese.
If you like to use Latin, call them: \emph{litera communis or litera officinalis} (common or business Chinese); \emph{litera classica minor} (lesser classical Chinese); and \emph{litera classica major} (higher classical Chinese.)

Now many foreigners have called themselves or have been called Chinese scholars.
Writing an article on Chinese scholarship, some thirty years ago for the \emph{N.C. Daily News}, --- ah me! those old Shanghai days, \emph{Tempora mutantur, nos et mutamur in illis}, \marginpar{\scriptsize Tempora mutantur, nos et mutamur in illis: Times change and we change with them} --- I then said: ``Among Europeans in China, the publication of a few dialogues in some provincial \emph{patois} or the collection of a hundred Chinese proverbs at once entitles a man to call himself a Chinese scholar.''
``There is,'' I said, ``of course no harm in a name, and with the extraterritoriality clause in the treaty, an Englishman in China may with impunity call himself Confucius, if so it pleases him.''
Now what I want to say here is this: how many foreigners who call themselves Chinese scholars, have any idea of what an asset of civilisation is stored up in that portion of Chinese literature which I have\footnote{Have in the original book} called the \emph{Classica majora}, the literature in full court dress Chinese?
I say an asset of civilisation, because I believe that this \emph{Classica majora} in the Chinese literature is something which can, as Matthew Arnold says of Homer's poetry, ``refine the raw natural man: they can transmute him.''
In fact, I believe this \emph{Classica majora} in Chinese literature will be able to transform one day even the raw natural men who are now fighting in Europe as patriots, but with the fighting instincts of wild animals; transform them into peaceful, gentle and civil persons.
Now the object of civilisation, as Ruskin says, is to make mankind into civil persons who will do away with coarseness, violence brutality and fighting.

But \emph{revenons \`a nos moutons}. \marginpar{\scriptsize revenons \`a nos moutons: back on topic} 
Is then written Chinese a difficult language?
My answer again is, yes and no.
I say, written Chinese, even what I have called the full court dress Chinese, the \emph{classica majora} Chinese, is not difficult, because, like the spoken or colloquial Chinese, it is extremely simple.
Allow me to show you by an average specimen taken at random how extremely simple, written Chinese even when dressed in full court dress uniform, is.
The specimen I take is a poem of four lines from the poetry of the T'ang dynasty \marginpar{\scriptsize T'ang dynasty (AD 618-907)} describing what sacrifices the Chinese people had to make in order to protect their civilisation against the wild half civilized fierce Huns \marginpar{\scriptsize Huns: 匈奴} from the North.
The words of the poem in Chinese are:
\begin{center}
    誓扫匈奴不顾身 \\
    五千貂锦丧胡尘 \\
    可怜无定河边骨 \\
    犹是春闺梦里人 \\
\end{center}
which translated into English word for word mean:
\begin{center}
    Swear sweep the Huns not care self, \\
    Five thousand embroidery sable perish desert dust; \\
    Alas! Wuting riverside bones, \\
    Still are Spring chambers dream inside men!
\end{center}
A free English version of the poem is something like this: ---\footnote{without --- in the original book}
\begin{center}
    They vowed to sweep the heathen hordes \\
    From off their native soil or die: \\
    Five thousand taselled knights, sable-clad, \\
    All dead now on the desert lie. \\
    Alas! the white bones that bleach cold \\
    Far off along the Wuting stream, \\
    Still come and go as living men \\
    Home somewhere in the loved one's dream.
\end{center}

Now, if you will compare it with my poor clumsy English version, you will see how plain in words and style, how simple in ideas, the original Chinese is.
How plain and simple in words, style and ideas: and yet how \emph{deep} in thought, how \emph{deep} in feeling it is.

In order to have an idea of this kind of Chinese literature, --- deep thought and deep feeling in\footnote{m in the original book} extremely simple language, --- you will have to read the Hebrew Bible. \marginpar{\scriptsize Hebrew Bible: 希伯来圣经}
The Hebrew Bible is one of the deepest books in all the literature of the world and yet how plain and simple in language.
Take this passage for instance: ``How is this faithful city become a harlot!
The men in the\footnote{the is missing in the original book} highest places are disloyal traitors and companions of thieves;
every one loveth\footnote{loves in the original book} gifts and followeth\footnote{follows in the original book} after rewards;
they judge not the fatherless neither doth\footnote{does in the original book} the cause of the widow come before them.'' 
(Is. I 21-23), or this other passage from the same prophet: --- ``I will make children to be their high officials and babes shall rule over them. And the people shall be oppressed. The child shall behave himself proudly against the old man and the base against the honourable!''
What a picture! The picture of the awful state of a nation or people.
Do you see the picture before you now?
In fact, if you want to have literature which can transmute men, can civilise mankind, you will have to go to the literature of the Hebrew people or of the Greeks or to Chinese literature.
But Hebrew and Greek are now become dead languages, whereas Chinese is a living language --- the language of four hundred million people still living today.

But now to sum up what I want to say on the Chinese language.
Spoken as well as written Chinese is, in one sense, a very difficult language.
It is difficult, not because it is complex.
Many European languages such as Latin and French are difficult because they are complex and have many rules.
Chinese is difficult not because it is complex, but because it is \emph{deep}.
It is difficult because it is a language for expressing deep feeling in simple language.
That is the secret of the difficulty of the Chinese language.
In fact, as I have said else where, Chinese is a language of the heart: a poetical language.
That is the reason why even a simple letter in prose written in classical Chinese reads like poetry.
In order to understand written Chinese, especially what I call full court dress Chinese, you must have your full nature, --- the heart and the head, the soul and the intellect equally developed.

It is for this reason that for people with modern European education, Chinese is especially difficult, because modern\footnote{modem in the original book} European education develops principally only one part of a man's nature --- his intellect.
In other words, Chinese is difficult to a man with modern\footnote{modem in the original book} European education, because Chinese is a deep language and modern European education, which aims more at quantity than quality, is apt to make a man \emph{shallow}.
Finally for half educated people, even the spoken language, as I have said, is difficult.
For half educated people it may be said of them as was once said of rich men, it is easier for a camel to go through the eye of a needle, than for them to understand high classical Chinese and for this reason: written Chinese is a language only for the use of really educated people.
In short, written Chinese, classical Chinese is difficult because it is the language of \emph{really educated} people and real education is a difficult thing but as the Greek proverb says, ``all beautiful things are difficult.''

But before I conclude, let me here give another specimen of written Chinese to illustrate what I mean by simplicity and depth of feeling which is to be found even in the \emph{Classica Minora}, literature written in official uniform Chinese.
It is a poem of four lines by a modern\footnote{modem in the original book} poet written on New year's\footnote{Year's in the original book} eve\footnote{Eve in the original book}.
The words in Chinese are: ---
\begin{center}
    示内 \\
    莫道家贫卒岁难 \\
    北风曾过几番寒 \\
    明年桃柳堂前树 \\
    还汝春光满眼看 \\
\end{center}
which, translated word for word, mean: ---
\begin{center}
    Don't say home poor pass year hard, \\
    North wind has blown many times cold, \\
    Next year peach willow hall front trees \\
    Pay-back you spring light full eyes see. \\
\end{center}
A free translation would be something like this:\footnote{? in the original book}
\begin{center}
    TO MY WIFE.\footnote{without . in the original book} \\
    Fret not, --- though poor we yet can pass the year; \\
    Let the north wind blow ne'er so chill and drear, \\
    Next year when peach and willow are in bloom, \\
    You'll yet see Spring and sunlight in our home.
\end{center}

Here is another specimen longer and more sustained.
It is a poem by Tu Fu, \marginpar{\scriptsize Tu Fu: 杜甫} the Wordsworth of China, of the T'ang\footnote{Tang in the original book} Dynasty.
I will here first give my English translation.
The subject is
\begin{center}
    MEETING WITH AN OLD FRIEND.\footnote{without . in the original book} \\
   In life, friends seldom are brought near; \\
   Like stars, each one shines in its sphere. \\
   Tonight, oh! what a happy night! \\
   We sit beneath the same lamplight. \\
   Our youth and strength last but a day. \\
   You and I --- ah! our hairs are grey. \\
   Friends! Half are in a better land, \\
   With tears we grasp each other's hand. \\
   Twenty more years, --- short, after all, \\
   I once again ascend your hall. \\
   When we met, you had not a wife; \\
   Now you have children, --- such is life! \\
   Beaming, they greet their father's chum; \\
   They ask me from where I have come. \\
   Before our say, we each have said, \\
   The table is already laid. \\
   Fresh salads from the garden near, \\
   Rice mixed with millet, --- frugal cheer. \\
   When shall we meet? 'tis hard to know. \\
   And so let the wine freely flow. \\
   This wine, I know, will do no harm. \\
   My old friend's welcome is so warm. \\
   Tomorrow I go, --- to be whirled. \\
   Again into the wide, wide world.
\end{center}

The above, my version I admit, is almost doggerel, which is meant merely to give the meaning of the Chinese text.
But here is the Chinese text which is not doggerel, but \emph{poetry} --- poetry simple to the verge of colloquialism, yet with a grace, dignity pathos and \emph{nobleness} which I cannot reproduce and which perhaps it is impossible to reproduce, in English in such simple language.
\begin{center}
    人生不相见 \qquad 动如参与商  \\
    今夕复何夕 \qquad 共此灯烛光 \\
    少壮能几时 \qquad 须发各已苍 \\
    访旧半为鬼 \qquad 惊呼热中肠 \\
    焉知二十载 \qquad 重上君子堂 \\
    昔别君未婚 \qquad 儿女忽成行 \\
    怡然敬父执 \qquad 问我来何方 \\
    问答未及已 \qquad 儿女罗酒浆 \\
    夜雨剪春韭 \qquad 新炊间黄粱 \\
    主称会面难 \qquad 一举累十觞 \\
    十觞亦不醉 \qquad 感君故意长 \\
    明日隔山岳 \qquad 世事两茫茫 \\
\end{center}

