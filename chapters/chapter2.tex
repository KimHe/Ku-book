\chapter{The Chinese Woman}
Matthew Arnold, \marginpar{\scriptsize Matthew Arnold: (1822-1888) an English poet and culture critic.} speaking of the argument taken from the Bible which was used in the House of Commons \marginpar{\scriptsize House of Commons is the lower house of the Parliament of the UK.} to support the Bill for enabling a man to marry his deceased wife's sister, said: ``Who will believe when he really considers the matter, that when the feminine nature, the feminine ideal and our relations with them are brought into question, the delicate and apprehensive genius of the Indo-European race, the race which invented the Muses, and Chivalry, and the Madonna, is to find its last word on this question in the institution of a Semitic people whose wisest King had seven hundred wives and three hundred concubines?''

The two words I want for my purpose here from the above long quotation are the words ``feminine ideal." 
Now what is the Chinese feminine ideal?
What is the Chinese people's ideal of the feminine nature and their relations to that ideal?
But before going further, let me, with all deference to Matthew Arnold, and respect for his Indo-European\footnote{Indo- European in the original book} race, say here that the feminine ideal of the Semitic race, of the old Hebrew people is not such a horrid one as Matthew Arnold would have us infer from the fact that their wisest King had a multitude of wives and concubines.
For here is the feminine ideal of the old Hebrew people, as we find it in their literature: ``Who can find a virtuous woman? for her price is far above rubies.
The heart of her husband doth safely trust in her.
She rises also while it is yet night and giveth meat to her household and a portion to her maidens.
She layeth her hands to the spindle and her fingers hold the distaff.
She is not afraid of snow for her household; for \emph{all her household are clothed in scarlet}.
She openeth her mouth with wisdom and \emph{in her tongue is the law of kindness}.
She looketh well to the ways of her household and eateth not the bread of idleness.
Her children rise up and call her blessed, her husband also and he praiseth her.''

This, I think, is not such a horrid, not such a bad ideal after all,--- this feminine ideal  of the Semitic race.
It is of course not so etherial\marginpar{\scriptsize etherial [sic: ethereal?]} as the Madonna and the Muses, the feminine ideal of the Indo-European\footnote{In- do-European in the original book} race.
However, one must, I think, admit, --- the Madonna and the Muses are very well to hang up as pictures in one's room, but if you put a broom into the hands of the Muses or send your Madonna into the kitchen, you will be sure to have your rooms in a mess and you will probably get in the morning no breakfast at all.
Confucius says, ``The ideal is not away from the actuality of human life.
When men take something away from the actuality of human life as the ideal, --- that is not the true ideal."\marginpar{\scriptsize 中庸 The Universal Order XIII}
But if the Hebrew feminine ideal cannot be compared with the Madonna and the Muses, it can very well, I think, compare with the modern\footnote{modem in the original book} European feminine ideal, the feminine ideal of the Indo-European race in Europe and America today.
I will not speak of the suffragettes in England.
But compare the old Hebrew feminine ideal with the modern\footnote{modem in the original book} feminine ideal such as one finds it in modern\footnote{modem in the original book} novels, with the heroine, for instance of Dumas' \marginpar{\scriptsize Alexandre Dumas (fils): (1824-1895) a French author and playwright; best known for the romantic novel \emph{La Dame aux Cam\'elias}. 小仲马} \emph{Dame aux Camelias}. \marginpar{\scriptsize Camelias [sic: Cam\'elias]}
By the way, it may interest people to know that of all the books in European literature which have been translated into Chinese, the novel of Dumas with the Madonna of the Mud as the superlative feminine ideal has had the greatest sale and success in the present up-to-date modern China.
This French novel called in Chinese the \emph{Cha-hua-nu} (茶花女) has even been dramatised and put on the stage in all the up-to-date\footnote{up-to- date in the original book} Chinese theatres in China.
Now if you will compare the old feminine ideal of the Semitic race, the woman who is not afraid of the snow for her household, for she has clothed them all in scarlet, with the feminine ideal of the Indo European race in Europe today, the Camelia Lady who has no\footnote{do in the original book} household, and therefore clotheth not her household, but herself in scarlet and goes with a Camelia flower on her breast to be photographed: then you will understand what is true and what is false, tinsel civilisation.

Nay, even if you will compare the old Hebrew feminine ideal, the woman who layeth her hands to the spindle and whose fingers hold the distaff, who looketh well to the ways of her household and eateth not the bread of idleness, with the up-to-date modern\footnote{modem in the original book} Chinese woman who layeth her hands on the piano and whose fingers hold a big bouquet, who, dressed in tight fitting yellow dress with a band of tinsel gold around her head, goes to show herself and sing before a miscellaneous crowd in the Confucian Association Hall: if you compare these two feminine ideals, you will then know how fast and far modern\footnote{modem in the original book} China is drifting away from true civilisation.
For the womanhood\footnote{woman- hood in the original book} in a nation is the flower of the civilisation, of the state of civilisation in that nation.

But now to come to our question: what is the Chinese feminine ideal? 
The Chinese feminine ideal I answer, is essentially the same as the old Hebrew feminine ideal with one important difference of which I will speak later on.
The Chinese feminine ideal is the same as the old Hebrew ideal in that it is not an ideal merely for hanging up as a picture in one's room; nor an ideal for a man to spend his whole life in caressing and worshipping.
The Chinese feminine ideal is an ideal with a broom in her hands to sweep and clean the rooms with.
In fact the Chinese written character for a wife (婦) is composed of two radicals --- (女) meaning a woman and (帚) meaning a broom.
In classical Chinese, in what I have called the official uniform Chinese, a wife is called the Keeper of the Provision Room --- a Mistress of the Kitchen (主中馈). \marginpar{\scriptsize 中馈:指家中供膳诸事}
Indeed the true feminine ideal, --- the feminine ideal of all people with a true, not tinsel civilisation, such as the old Hebrews, the ancient Greeks and the Romans, is essentially the same as the Chinese feminine ideal: the true feminine ideal is always the \emph{Hausfrau}\marginpar{\scriptsize Hausfrau 家庭主妇(德语)}, the house wife, \emph{la\footnote{La in the original book} dame de menage \marginpar{\scriptsize menage [sic: m\'enage] \\ la dame de m\'enage: the housekeeper} or chatelaine}.

But now to go more into details.
The Chinese feminine ideal, as it is handed down from the earliest times, is summed up in three obediences\footnote{Three Obediences in the original book} (三从) and Four Virtues (四德). 
Now what are the four virtues?
They are: first womanly character (女德); second, womanly conversation (女言); third, womanly appearance (女容); and lastly, womanly work (女工).
Womanly character means not extraordinary talents or intelligence, but modesty, cheerfulness, chastity, constancy, orderliness, blameless conduct and perfect manners.
Womanly conversation means not eloquence or brilliant talk, but refined choice of words, never to use coarse or violent language, to know when to speak and when to stop speaking.
Womanly appearance means not beauty or prettiness of face, but personal cleanliness and faultlessness in dress and attire.
Lastly\footnote{, lastly in the original book}, womanly work means not any special skill or ability, but assiduous attention to the spinning room, never to waste time in laughing and giggling and work in the kitchen to prepare clean and wholesome food, especially when there are guests in the house.
These are the four\footnote{lout in the original book} essentials in the conduct of a woman as laid down in the " Lessons for Woman" (女诫), written by Ts'ao Ta Ku (曹大家) \marginpar{\scriptsize 曹大家:(45-117)东汉女史学家、文学家} or Lady Ts'ao, sister of the great historian Pan Ku (班固) of the Han Dynasty.

Then again what do the Three Obediences (三从) in the Chinese feminine ideal mean?
They mean really three self sacrifices or ``live for's?"
That is to say, when a woman is unmarried, she is to live for her father (在家从父); when married, she is to live for her husband (出嫁从夫); and, as a widow, she is to live for her children (夫死从子).
In fact, the chief end of a woman in China is not to live for herself, or for society; not to be a reformer or to be president of the woman's natural feet Society; not to live even as a saint or to do good to the world; the chief end of a woman in China is to live as a good daughter a good wife and a good mother.

A foreign lady friend of mine once wrote and asked me whether it is true that we Chinese believe, like\footnote{tike in the original book} the Mohamedans, that a woman has no soul.
I wrote back and told her that we Chinese do not hold that a woman has no soul, but that we hold that a woman, --- a true Chinese woman has no \emph{self}.
Now speaking of this ``no self'' in the Chinese woman leads me to say a few words on a very difficult subject, --- a subject which is not only difficult, but, I am afraid almost impossible for people with the modern\footnote{modem in the original book} European education to understand, viz. concubinage in China.
This subject of concubinage, I am afraid, is not only a difficult, but also a dangerous subject to discuss\footnote{dis- cuss in the original book} in public.
But, as the English poet says.
\begin{quote}\footnotesize
     Thus fools rush in where angels fear to tread,
 \end{quote}

I will try my best here to explain why concubinage in China is not such an immoral custom as people generally imagine.

The first thing I want to say on this subject of concubinage is that it is the selflessness in the Chinese woman which makes concubinage in China not only possible, but also \emph{not\footnote{no in the original book} immoral}.
But, before I go further, let me tell you here, that concubinage in China does not mean having many \emph{wives}.\footnote{{\huge$\cdot$} in the original book} 
By Law in China, a man is allowed to have only \emph{one} wife, but he may have as many handmaids or concubines as he like. \marginpar{\scriptsize like [sic: likes]}
In Japanese a handmaid or concubine is called \emph{te-kaki}\footnote{tekaki in the original book}, a hand rack or \emph{me-kaki} an eye rack; --- \ie\, to say, a rack where to rest your hands or eyes on when you are tired.
Now the feminine ideal in China, I said\footnote{Said in the original book}, is not an ideal for a man to spend his whole life in caressing and worshipping.
The Chinese feminine ideal is, for a wife to live absolutely, selflessly for her husband.
Therefore when a husband who is sick or invalided from over work\footnote{overwork in the original book} with his brain and mind, requires a handmaid, a hand rack or eye rack to enable him to get well and to fit\footnote{\{it in the original book} him for his life work, the wife in China with her selflessness, gives it to him just as a good wife in Europe and America gives an arm chair\footnote{armchair in the original book} or goat's milk to her husband when he is sick or requires it.
In fact it is the selflessness of the wife in China, her sense of duty, the duty of self sacrifice which allows a man in China to have handmaids or concubines.

But people will say to me, ``why ask selflessness and sacrifice only from the woman? What about the man?'' 
To this, I answer, does not the man, --- the husband, who toils and moils to support his family, and especially if he is a gentleman, who has to do his duty not only to his family, but to his King and country, and, in doing that has, some time even to give his life: does he not also make sacrifice?
The Emperor Kanghsi\marginpar{\scriptsize Kanghsi: 康熙} in a valedictory decree which he issued on his death bed, said that ``he did not know until then what a life of sacrifice the life of an Emperor in China is." 
And yet, let me say here by the way, Messrs.~J.~B.~Bland \marginpar{\scriptsize J.~B.~Bland [sic: J.~P.~Bland] \\ John Otway Percy Bland: (1863-1945) a British writer and journalist.} and Backhouse \marginpar{\scriptsize Edmund Trelawny Backhouse: (1873-1944) a British oriental scholar, sinologist and linguist.} in their latest book have described this Emperor Kanghsi as a huge, helpless, horrid Brigham Young, \marginpar{\scriptsize Brigham Young: (1801-1877) an American religious leader, politian and settler.} who was dragged into his grave by the multitude of his wives and children.
But, of course, for modern men like Messrs.~J.~P.~Bland\footnote{Messrs.~J-P.~Bland in the original book} and Backhouse, concubinage is  inconceivable except as something horrid, vile and nasty, because the diseased imagination of such men can conceive of nothing except nasty, vile and horrid things.
But that is neither here nor there.
Now what I want to say here is that  the life of every \emph{true} man --- from the Emperor down to the ricksha coolie \marginpar{\scriptsize ricksha coolie: 黄包车苦力} --- and every \emph{true} woman, is a life of sacrifice.
The sacrifice of a woman in China is to live selflessly for the man whom she calls husband, and the sacrifice of the man in China is to provide for, to protect at all costs the woman or women whom he has taken into his house and also the children they may bear him.
Indeed to people who talk of the immorality of concubinage in China, I would say that to me the Chinese mandarin who keeps concubines is less selfish, less immoral than the European in his motor car, who picks up a helpless woman from the public street and, after amusing himself with her for one night, throws her away again on the pavement of the public street the next morning.
The Chinese mandarin with his concubines may be selfish, but he at least provides a house for his concubines and holds himself for life responsible for the maintenance of the women he keeps.
In fact, if the mandarin is selfish, I say that the European in his motor car is not only selfish, but a \emph{coward}.\footnote{{\huge$\cdot$} in the original book}
Ruskin \marginpar{\scriptsize John Ruskin: (1819-1900) was the leading English art critic of the Victorian era.} says, ``The honour of a true soldier is verily not to be able to slay, but to be willing and ready at all times to \emph{be slain}.\footnote{{\huge$\cdot$} in the original book}"
In the same way I say, the honour of a woman --- a true woman in China, is not only to love and be true to her husband, but to live absolutely, selflessly for him.
In fact, this Religion of Selflessness is the religion of the woman, especially, the gentlewoman or lady in China, as the Religion of Loyalty which I have tried elsewhere to explain, is the religion of the man,--- the gentleman in China.
Until foreigners come to understand these two religions, the ``Religion of Loyalty and the Religion of Selflessness" of the Chinese people, they can never understand the real Chinaman, or the real Chinese woman.

But people will again say to me, ``What about love? Can a man who really loves his wife have the heart to have other women besides her in his house?"
To this I answer, yes, --- Why not? 
For the real test that a husband really loves his wife is not that he should spend his whole life in lying down at her feet and caressing her.
The real test whether a man truly loves his wife is whether he is anxious and tries in every thing reasonable, not only to protect her, but also not to hurt her, not to hurt her feelings.
Now to bring a strange woman into the house must hurt the wife, hurt her feelings.
But here, I say, it is what I have called the Religion of selflessness which protects the wife from being hurt: it is this absolute selflessness in the woman in China which makes it possible for her not to feel hurt when she sees her husband bring another woman into the house.
In other words, it is the selflessness in the wife in China which enables, \emph{permits} the husband to take a concubine without hurting the wife.
For here, let me point out, a gentleman, --- a real gentleman in China, never takes a concubine without the consent of his wife and a real gentlewoman or lady in China whenever there is a proper reason that her husband should take a concubine, will never refuse to give her consent.
I know of many cases where having no children the husband after middle age wanted take a concubine, but because the wife refused to give her consent, desisted.
I know even of a case where the husband, because he did not want to exact this mark of selflessness from his wife who was sick and in bad health, refused, when urged by the wife, to take a concubine, but the wife, without his knowledge and consent, not only bought a concubine, but actually forced him to take the concubine into the house.
In fact, the protection for the wife against the abuse of concubinage in China is the \emph{love of her husband for her}.
Instead, therefore of saying that husbands in China cannot truly love their wives because they take concubines, one should rather say it is because the husband in China so \emph{truly} loves his wife that he has the privilege and liberty of taking concubines without fear of his abusing that privilege and liberty.
This liberty, this privilege is sometimes and even --- when the sense of honour in the men in the nation is low as now in this anarchic China, often abused.
But still I say the protection for the wife in China where the husband is allowed to take a concubine, is the love of her husband for her, the love of her husband, and, I must add here, his \emph{tact} --- the perfect good taste in the real Chinese gentleman.
I wonder if one man in a thousand among the ordinary Europeans and Americans, who can keep more than one woman in the same house without turning the house into a fighting cockpit or hell.
In short, it is this tact, --- the perfect good taste in the real Chinese gentleman which makes it possible for the wife in China not to feel hurt, when the husband takes and keeps a handmaid, a hand rack, an eye rack in the same house with her.
But to sum up, --- it is the Religion of selflessness, the absolute selflessness of the woman, --- the gentlewoman or lady and the love of the husband for his wife and his tact, --- the perfect good taste of a real Chinese gentleman, which, as I said, makes concubinage in China, not only possible, but also \emph{not immoral}.\footnote{, in the original book} Confucius said, ``The Law of the Gentleman takes its rise from the relation between the husband and the wife."

Now in order to convince those who might still be sceptical that husbands in China \emph{truly} love, can \emph{deeply} love their wives, I could produce abundant proofs from Chinese history and literature.
For this purpose I should particularly like to quote and translate here an elegy written on the death of his wife by Yuan Chen (元稹), a poet of the T'ang\footnote{Tang in the original book} dynasty.
But unfortunately the piece is too long for quotation here in this already too long\footnote{long is missing in the original book} article.
Those acquainted with Chinese, however, who wish to know how deep the affection, ---\footnote{without --- in the original book} affection, true love and not sexual passion which in modern\footnote{modem in the original book} times is often mistaken for love, --- how deep the love of a husband in China for his wife is, should read this elegy which can be found in any ordinary collection of the T'ang\footnote{Tang in the original book} poets.
The title of the elegy is, (遣悲怀) --- ``Lines to ease the aching heart." 
But as I cannot use this elegy for my purpose, I will, instead, give here a short poem of four lines written by a modern\footnote{modem in the original book} poet who was once a secretary of the late Viceroy Chang Chih-tung. \marginpar{\scriptsize Chang chih-tung: 张之洞}
The poet went together with his wife in the suite of the Viceroy to Wuchang and after staying there  many years, his wife died.
Immediately after he too had to leave Wuchang.
He wrote the poem on leaving Wuchang.\footnote{with --- in the original book} 
The words in Chinese are
\begin{quote}\footnotesize
   此恨人人有\\
   百年能有幾\\
   痛哉长江水\\
   同渡不同歸
\end{quote}
The meaning in English is something like this: --- 
\begin{quote}\footnotesize
    This grief is common to everyone,\footnote{. in in the original book}\\
    One hundred years how many can attain?\\
    But 'tis heart breaking, o waters of the Yangtze,\\
    Together we came, --- but together we return not.\\
\end{quote}
The feeling here is as deep, if not deeper; but the words are fewer, and the language is simpler, even than Tennyson's.
\begin{quote}\footnotesize
    Break, break, break\\
    On thy\footnote{the in the original book} cold grey stones, O sea!\\
               ......              \\
    But O for the touch of a vanished hand,\\
    And the sound of a voice that is still!
\end{quote}

But now what about the love of a wife in China for her husband?
I do not think any evidence is needed to prove this.
It is true that in China the bride and bride-groom as a rule never see each other until the marriage day, and yet that there is love between even bride and bride-groom, can be seen in these four lines of poetry from the T'ang\footnote{Tang in the original book} dynasty: --- 
\begin{quote}\footnotesize
      洞房昨夜停紅燭\\
      待曉堂前拜舅姑\\
      妆罷低聲問夫婿\\
      畫眉深淺入時無
\end{quote}
The meaning in English of the above is something like this,
\begin{quote}\footnotesize
    In the bridal chamber last\footnote{Last in the original book} night stood red candles;\\
    Waiting for the morning to salute the father and mother in the hall,\\
    Toilet finished, --- in a low voice she asks her sweetheart\footnote{sweetheart in the original book} husband,\\
    ``Are the shades in my painted eyebrows quite \`a la mode?'' \marginpar{\scriptsize \`a la mode de: French, in the style of}
\end{quote}

But here in order to understand the above, I must tell you something about marriage in China.
There are in every legal marriage in China six ceremonies (六礼): first, (问名) asking for the name, \ie, formal proposal; second (纳彩) receiving the silk presents, \ie, betrothal: third (定期) fixing the day of marriage; fourth (亲迎) fetching the bride; fifth (奠雁) pouring libation \marginpar{\scriptsize libation: 祭酒} before the wild  goose, \ie, plighting troth, so-called because the wild goose is supposed to be most faithful in connubial love; sixth (庙见) --- temple presentation.
Of these six ceremonies, the last two are the most important, I shall therefore here describe them more in detail.

The fourth ceremony, fetching the bride at the present day, is, except in my province Fukien \marginpar{\scriptsize Fukien: 福建} where we keep up the old customs, --- generally dispensed with, as it entails too much trouble and expense to the bride's family.
The bride now, instead of being fetched, is sent to the bride-grooms'\footnote{bride-groom  $\^$ in the original book} house.
When the bride arrives there, the bridegroom receives her at the gate and himself opens the door of the bridal chair and leads her to the hall of the house.
There the bride and\footnote{and , in the original book} bride-groom worship Heaven and Earth (拜天地), \ie\, to say, they fall on their knees\footnote{kness in the original book} with their faces turned to the door of the hall with a table carrying two red burning candles before the open sky and then the husband pours libations on the ground, --- in presence of the pair of wild geese (if wild goose cannot be had, an ordinary goose) which the bride has brought with her.
This is the ceremony called \emph{Tien yen} pouring libation before the wild goose; plighting of troth between man and woman --- he vowing to be true to her, and she, to be true to him, just as faithful as the pair of wild geese they see before them.
From this moment, they become, so to speak, natural \emph{sweetheart husband} and \emph{sweetheart wife}, bound only by the moral law, the Law of the Gentleman, --- the word of honour which they have given to each other, but not yet by the Civic Law.
This ceremony therefore may be called the moral or Religious marriage.

After this comes the ceremony called the (交拜) mutual salutation between bride and bride-groom.
The bride standing on the right side of the hall first goes on her knees before the bride-groom, --- he going on his knees to her at the same time.
Then they change places.
The bride-groom now standing where the bride stood, \emph{goes on his knees to her}, --- she returning the salute just as he did.
Now this ceremony of \emph{chiao pai} mutual salutation, I wish to point out here, proves beyond all doubt that in China there is \emph{perfect equality} between man and woman, between husband and wife.

As I said before, the ceremony of plighting troth may be called the moral or Religious marriage as distinguished from what may called the \emph{civic} marriage, which comes three days after.
--- In the moral or religious marriage, the man and women \marginpar{\scriptsize women [sic: woman]} becomes husband and wife before the moral Law --- before God.
The contract so far is solely between the man and woman.
The State or, as, in China, the Family takes\footnote{lakes in the original book} the place of the State in all social and civic life --- the State acting only as Court of appeal, --- the Family takes no cognisance of the marriage or contract between the man and woman here in this, what I have called the moral or religious marriage.
In fact on this first day and until the \emph{civic} marriage takes place on the third day of the marriage, the bride is not only \emph{not} introduced, but also not allowed to see or be seen by the members of the bride-groom's\footnote{bridegroom's in the original book} family. 

Thus for two days and two nights the bride-groom and the bride in China live, so to speak not as legal, but, as \emph{sweetheart-husband} and \emph{sweet-heart wife}\footnote{sweetheart-wife in the original book}.
On the third day, --- then comes the last ceremony in the Chinese marriage --- the \emph{Miao-chien}, the temple presentation or civic marriage.
I say, on the third day because that is the rule \emph{de riguer} \marginpar{\scriptsize riguer [sic: rigueur] \\ \emph{de eigueur}: French, according to strict etiquette} as laid down in the Book of Rites (三日庙见). \marginpar{\scriptsize Book of Rites: 《礼记》}
But now to save trouble and expense, it is generally performed on the day after.
This ceremony --- the temple presentation, takes place, when the ancestral temple of the family clan is near by, --- of course in the ancestral temple.
But for people living in towns and cities where there is no ancestral temple of the family clan near by\footnote{nearby in the original book}, the ceremony is performed before the miniature ancestral chapel or shrine --- which is in the house of every respectable family, even the poorest in China.
This ancestral temple, chapel or shrine with a tablet or red piece of paper on the wall, as I have said elsewhere, is the \emph{church} of the State Religion of Confucius in China corresponding to the church of the Church Religion in Christian countries.

This ceremony --- the temple presentation begins by the father of the bride groom\footnote{bridegroom in the original book} \marginpar{\scriptsize bride groom [sic: bride-groom]} or failing him, the nearest senior member of the family, going on his knees before the ancestral tablet --- thus announcing to the spirits of the dead ancestors that a young member of the family has now brought a wife home into the family.
Then the bride groom\footnote{bridegroom in the original book} \marginpar{\scriptsize bride groom [sic: bride-groom]} and bride one after the other, each goes on his and her knees before the same ancestral tablet.
From this moment the man and woman becomes husband and wife, --- not only before the moral Law or God, --- but before the Family, before the State, before Civic Law.
I have therefore called this ceremony of \emph{miao chien}, temple presentation in the Chinese marriage, --- the civic or civil marriage.
Before this civic or civil marriage, the woman, the bride, --- according to the Book of Rites, --- is not a legal wife (不庙见不成妇).
When the bride happens to die before this ceremony of temple presentation, she is not allowed --- according to the Book of Rites --- to be buried in the family burying ground of her husband and her memorial tablet is not put up in the ancestral temple of his family clan.

Thus we see the contract in a legal civic marriage in China is not between the woman and the man.
The contract is between the woman and the family of her husband.
She is not married to him, but \emph{into his family}.
In the visiting card of a Chinese lady in China, she does not write, for instance, Mrs.~Ku Hung-ming, but literally ``Miss Feng, gone to the home of the family (originally from) Tsin An adjusts her dress,\footnote{. in the original book}" (歸晋安馮氏裣衽).
--- The contract of marriage in China \emph{being between the woman and the family of her husband}, --- the husband and wife can neither of them repudiate the contract without the consent of the husband's family.
This I want to point out here, is the fundamental difference between a marriage in China and a marriage in Europe and America.
The marriage in Europe and America, --- is what we Chinese would call a sweet-heart marriage, a marriage, bound solely by love between the individual man and the individual woman.
But in China the marriage is, as I have said, a civic marriage, a contract not between the woman and the man, \emph{but between the woman\footnote{'woman in the original book} and the family of her husband}, ---\footnote{without --- in the original book} in which she has obligations not only to him, but also to his family, and through the family, to society, --- to the social or civic order; in fact, to the State.
Finally let me point out here that it is this civic conception of marriage which gives solidarity and stability to the family, to the social or civic order, to the State in China. 
Until therefore, let me be permitted to say here, --- the people in Europe and America understand what true \emph{civic life} means, understand and have a true conception of what it is really to be a citizen, --- a citizen not each one living for himself, but each one living first for his family, and through that for the civic order or State, --- there can then be no such thing as a stable society, civic order or State in the true sense of the word.
--- A State such as we see it in modern\footnote{modem in the original book} Europe and American today\footnote{to-day in the original book}, where the men and women have not a true conception of civic life, --- such a State with all its parliament and machinery of government, may be called, if you like, --- a big Commercial Concern, or as it really is, in times of war, a gang of brigands and pirates, --- but not a State.
In fact, I may be permitted further to say here, it is the false conception of a State\footnote{Stale in the original book} as a big commercial concern having only the selfish material interests of those who have the biggest shares in the concern to be considered, --- this false conception of a State with the \emph{esprit de corps} \marginpar{\scriptsize esprit de corps: French, a feeling of pride and mutual loyalty shared by the members of a group} of brigands, which is, at bottom, the cause of the terrible war now going on in Europe.
In short, without a true conception of civic life there can be no true State and without a true State\footnote{state in the original book}, how can there be civilisation.
To us Chinese, a man who does not\footnote{t in the original book} marry, who has no family, no home which he has to defend, cannot be a patriot, and if calls himself a patriot, ---\footnote{without --- in the original book} we Chinese call him a \emph{brigand patriot}.
In fact in order to have a true conception of a State or civic order, one must first have a true conception of a family, and to have a true conception of a family, of family life, one must first of all have a true conception of marriage, --- marriage not as a sweetheart marriage, but as a civic marriage which I have in the above tried to describe.

But to return from the digression.
Now you can picture to yourself how the sweet-heart wife waiting for the morning --- to salute the father and mother of her husband, toilet finished, in a low voice, whispers to her sweet-heart husband and asks if her eyebrows are painted quite \`a la mode
--- Here you see, I say, there is love between husband and wife in China, although they have not seen each other before the marriage --- even on the third day of the marriage.
But if you think the love in the above is not deep enough, then take just these two lines of poetry from a wife to her absent husband.
\begin{quote}\footnotesize
   當君懷歸日\\
   是妾斷腸時\\
\end{quote}
\begin{quote}\footnotesize
   The day when you think of coming home.\\
   Ah! then my heart will already be broken.\\
\end{quote}

Roselind in Shakespeare's ``As you Like it"\footnote{As You Like It in the original book} \marginpar{\scriptsize A you Like it [sic: As You Like It]} says to her cousin Celia: ``O coz, coz, my pretty little coz, that thou knowest how many fathom deep I am in love!
But I\footnote{1 in the original book} cannot be sounded: my affection hath an unknown bottom, like the bay of Portugal."
Now the love of a woman, --- of a wife for her husband in China and also the love of the man --- of the husband for his wife in China, one can truly say, is like Roselind's\footnote{Rosolind's in the original book} love, many fathom deep and cannot be sounded; it has an unknown bottom like the bay of Portugal.

But,\footnote{; in the original book} I will now speak of the difference which, I said, there is between the Chinese feminine ideal and the feminine ideal of the old Hebrew people.
The Hebrew lover in the Songs of Solomon, thus addresses his lady-love: ``Thou art beautiful, O my love, as Tirzah, comely as Jerusalem, \marginpar{\scriptsize Jerusalem: 耶路撒冷} \emph{terrible as an army with banners}!"
People who have seen beautiful dark-eyed Jewesses even today\footnote{to day in the original book}, will acknowledge the truth and graphicness of the picture which the old Hebrew lover here gives of the feminine ideal of his race.
But in and about the Chinese feminine ideal, I want to say here, there is nothing \emph{terrible} either in a physical or in a moral sense.
Even the Helen of Chinese history, --- the beauty, who with one glance brings down a city and with another glance destroys a kingdom (一顾倾人城再顾倾人国) she is terrible only metaphorically\footnote{mataphorically in the original book}.
In an essay on ``the Spirit of the Chinese people\footnote{People in the original book},"
I said that the one word which will sum up the total impression which the Chinese type of humanity makes upon you is the English word, ``gentle."\footnote{Gentle'' in the original book}
If this is true of the real Chinaman, it is truer of the real Chinese woman.
In fact this ``gentleness" of the real Chinaman, in the Chinese woman, becomes sweet \emph{meekness}.
The meekness, the submissiveness of the woman in China is like that of Milton's \marginpar{\scriptsize John Milton: (1608-1674) an English poet, polemicist.} Eve in the ``Paradise Lost," \marginpar{\scriptsize Paradise Lost is an epic poem in blank verse by the 17th century English poet, John Miltion. 失乐园} who says to her husband,
\begin{quote}\footnotesize
   God is thy law, thou, mine; to know no more\\
   Is woman's happiest knowledge and her praise.\\
\end{quote}

Indeed this quality of perfect meekness in the Chinese feminine ideal you will find in the feminine ideal of no other people, --- of no other civilisation, Hebrew, Greek or Roman.
This perfect, \emph{divine} meekness in the Chinese feminine ideal you will find only in one civilisation, --- the Christian civilisation, when that civilisation in Europe reached its perfection, during the period of the \emph{Renaissance}.
If you will read the beautiful story of Griselda in Boccacio's\marginpar{\scriptsize Boccacio [sic: Boccaccio] \\ Giovanni Boccaccio: (1313-1375) an Italian writer, poet, correspondent of Petrarch.} \emph{Decameron} and see the true Christian feminine ideal shown there, you will then understand what this perfect submissiveness, this \emph{divine} meekness, meekness to the point of absolute selflessness, --- in the Chinese feminine ideal means.
In short, in this quality of divine meekness, the \emph{true} Christian feminine ideal is the Chinese feminine ideal, with just a shade of difference.
If you will\footnote{will is missing in the original book} carefully compare the picture of the Christian Madonna with, --- not the Buddhist Kuan Yin\marginpar{\scriptsize Kuan Yin: 观音}, --- but with the pictures of women fairies and genii painted by famous Chinese artists, you will be able to see this difference, --- the difference between the Christian feminine ideal, and the Chinese feminine ideal.
The Christian Madonna is meek and so is the Chinese feminine ideal.
The Christian Madonna is etherial\marginpar{\scriptsize etherial [sic: ethereal?]} and so is the Chinese feminine ideal.
But the Chinese feminine ideal is more than all that; the Chinese feminine ideal is \emph{debonair}.
To have a conception of what this charm and grace expressed by the word \emph{debonair} mean, you will have to go to ancient Greece, ---\footnote{without --- in the original book}
\begin{quote}\footnotesize
   \emph {\hfill o ubi campi\\ Spercheosque et virginibus bacchata Lacaenis Taygeta! \hfill}
\end{quote}

In fact you will have to go to the fields of Thessaly \marginpar{\scriptsize Thessaly: modern administrative region of Greece.} and the streams of Spercheios, \marginpar{\scriptsize Spercheios: a river in central Greece} to the hills alive with the dances of the Laconian \marginpar{\scriptsize Laconia: a region in the southeastern part of the Peloponnese peninsula.} maidens, --- the hills of Taygetus. \marginpar{\scriptsize Taygetus: a mountain range in the Peloponnese peninsula.}

Indeed I want to say here that even now in China since the period of the Sung\footnote{Song in the original book} Dynasty (A.D. 960), when what  may be called the Confucian Puritanism of the Sung philosophers \marginpar{\scriptsize Confucian Puritanism of the Sung philosophers: 宋朝理学} has narrowed, petrified, and in a way, \emph{vulgarised} the spirit of Confucianism, the spirit of the Chinese civilisation --- since then, the womanhood in China has lost much of the grace and charm, --- expressed by the word \emph{debonair}.
Therefore if you want to see the grace and charm expressed by the word debonair in the true Chinese feminine ideal, you will have to go to Japan where the women there at least, even to this day, have preserved the pure Chinese civilisation of the T'ang\footnote{Tang in the original book} Dynasty.
It is this grace and charm expressed by the word debonair combined with the \emph{divine meekness} of the Chinese feminine ideal, which gives the air of \emph{distinction} (名贵) to the \emph{Japanese} woman, --- even to the \emph{poorest} Japanese woman today.

In connection with this quality of charm and grace expressed by the word debonair, allow me to quote to you here a few words from Matthew Arnold with which he contrasts the \emph{brick-and-mortar} \marginpar{\scriptsize brick-and-mortar:?} Protestant English feminine ideal with the delicate Catholic French feminine ideal.
Comparing Eug\'enie de Guerin, \marginpar{\scriptsize Guerin [sic: Gu\'erin] \\ Eug\'enie de Gu\'erin: (1805-1848) a French writer and the sister of the poet Maurice de Gu\'erin.} the beloved sister of the French poet Maurice de Guerin, with an English woman who wrote poetry, Miss Emma Tatham, \marginpar{\scriptsize Emma Tatham: (1829-1855) a 19th-century English poet.} --- Matthew Arnold says: ``The French woman is a Catholic in Languedoc; the English woman is a Protestant at Margate, Margate the brick and mortar image of English Protestantism, representing it in all its prose, all its uncomeliness, --- and let me add, all its salubrity.
Between the external form and fashion of these two lives, between the Catholic Madlle de Guerin's \emph{nadalet} at the Languedoc Christmas, her chapel of moss at Easter time, her daily reading of the life of a saint, --- between all this and the bare, blank, narrowly English setting of Miss Tatham's Protestantism, her ``union in Church fellowship with the worshippers at Hawley Square, Margate,'' her singing with the soft, sweet voice, the animating lines.\footnote{: in the original book}
\begin{quote}\footnotesize
   My Jesus to know, and feel His Blood flow\\
   `Tis life everlasting, `tis heaven below!''\\
\end{quote}
her young female teachers belonging to the Sunday school and her ``Mr.~Tho\-mas Rowe, a venerable class-leader" --- what a dissimilarity.
In the ground of the two lives, a likeness; in all their circumstances, what unlikeness!
An unlikeness, it will be said, in that which is non-essential and indifferent.
Non-essential, --- yes; indifferent, --- no.
The signal \emph{want of grace and charm} --- in the English Protestantism's setting of its religious life is not an indifferent matter; it is a real weakness.
\emph{This ought ye to have done, and not to have left the other undone}."

Last of all I wish to point out to you here the most important quality of all, in the Chinese feminine ideal, the quality which prominently\footnote{preeminently in the original book} distinguishes her from the feminine ideal of all other people or nations ancient or modern\footnote{modem in the original book}.
This quality in the women in China, it is true, is common to the feminine ideal of every people or nation with any pretension to civilisation, but this quality, I want to say here, developed in the Chinese feminine ideal to such a degree of perfection as you will find it nowhere else in the world.
This quality of which I speak, is described by the two Chinese words \emph{yu hsien} (幽闲) which, in the quotation I gave above from the ``Lessons for Women," by Lady T'sao, --- I translated as modesty and cheerfulness.
The Chinese word \emph{yu} (幽) literally means retired, secluded, occult and the word \emph{hsien} (闲) literally means ``at ease or leisure."
For the Chinese word \emph{yu}, --- the English ``modesty, bashfulness" only gives you an idea of its meaning.
The German word \emph{Sittsamkeit} comes nearer to it.
But perhaps the French \emph{pudeur} comes nearest to it of all.
This \emph{pudeur}, I may say here, this bashfulness, the quality expressed by the Chinese word \emph{yu} (幽) is the essence of all\footnote{ail in the original book} womanly qualities.
The more a woman has this quality of \emph{pudeur} developed in her, the more she has of womanliness, --- of femininity, in fact, the more she is a perfect or ideal woman.
When on the contrary a woman loses this quality expressed by the Chinese word \emph{yu} (幽) , loses this bashfulness, this \emph{pudeur}, she then loses altogether her womanliness, her femininity, and with that, her perfume, her fragrance and becomes a mere piece of human meat or flesh.
Thus, it is this \emph{pudeur}, this quality expressed by the Chinese word \emph{yu}\footnote{, in the original book} (幽) in the Chinese feminine ideal which makes or \emph{ought} to make every \emph{true} Chinese woman instinctively feel and know that it is wrong to show herself in public; 
that it is \emph{indecent}, according to the Chinese idea, to go on a platform and sing before a crowd in the hall even of the Confucian Association.
In fine, it is this \emph{yu hsien} (幽闲), this love of seclusion, this sensitiveness against the ``garish eye of day;" this \emph{pudeur} in the Chinese feminine ideal, which gives to the true Chinese woman in China as to no other woman in the world, --- a perfume, a perfume sweeter than the perfume of violets, the ineffable fragrance of orchids.

In the oldest Love\footnote{love in the original book} song, I believe, of the world, which I translated for the \emph{Peking Daily News} two years ago --- the first piece in the \emph{Shih Ching}\marginpar{\scriptsize Shih Ching 诗经} or Book of Poetry, the Chinese feminine ideal is thus described,
\begin{quote}\footnotesize
   The birds are calling in the air, ---\\
   An islet by the river-side\footnote{river side in the original book};\\
   The maid is meek and debonair,\\
   Oh! Fit to be our Prince's bride.\\
\end{quote}

The words \emph{yao t'iao} (窈窕) have the same signification as the\footnote{the is missing in the original book} words \emph{yu hsien} (幽闲) meaning literally \emph{yao} (窈) secluded, meek, shy, and \emph{t'iao} (窕) attractive, debonair, and the words \emph{shu nu} (淑女) mean a pure, chaste girl or woman.
Thus here in the oldest love song in China, you have the three essential qualities in the Chinese feminine ideal, viz. love of seclusion, bashfulness or \emph{pudeur}, ineffable grace and charm expressed by the word debonair and last of all, purity or chastity.
In short, the real or true Chinese woman is chaste; she is bashful, has \emph{pudeur}; and she is attractive and debonair.
This then is the Chinese feminine ideal,--- the ``Chinese Woman."

In the Confucian Catechism (中庸) which I have translated as the Conduct of Life, the first part of the book containing the practical teaching of Confucius on the conduct of life concludes with the description of a Happy Home thus:
\begin{quote}\footnotesize
    ``When wife\footnote{-wife in the original book} and children dwell in unison,\\
    'Tis like to harp and lute-well played\footnote{lute well-played in the original book} in tune,\\
    When brothers live in concord and in peace,\footnote{. in the original book}\\
   The strain of harmony shall never cease.\\
   Make then your Home thus always gay and bright.\\
   Your wife and dear ones shall be your delight.''\\
\end{quote}
This Home in China is the miniature Heaven,--- as the State with its civic order, the Chinese Empire, --- is the real Heaven, the Kingdom of God come upon this earth, to the Chinese people.
Thus, as the gentleman in China with his honour, his Religion of Loyalty is the guardian of the \emph{State}, the Civic Order, in China, so the Chinese woman, the Chinese gentlewoman or lady, with her debonair charm and grace, her purity, her pudeur, and above all, her Religion of Selflessness, --- is the Guardian Angel of the miniature Heaven, the \emph{Home} in China.         

