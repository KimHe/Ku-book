\chapter{John Smith in China}
{ \scriptsize ``The Philistine not only ignores all conditions of life which are not his own but he also demands that the rest of mankind should fashion its mode of existence after his own.''\cite{num24} \dots GOETHE.}

Mr.~W.~Stead once asked: ``What is\footnote{Is in the original book} the secret of Marie Corelli's popularity?'' His answer was: ``Like author, like reader; because the John Smiths who read her novels live in Marie Corelli's world and regard her as the most authoritative exponent of the Universe in which they live, move and have their being.''
What Marie\footnote{Mane in the original book} Corelli is to the John Smiths in Great Britain, the Rev. Arthur Smith\marginpar{\scriptsize Arthur Henderson Smith: (1845-1932) \\ the author of Chinese Characteristics} is to the John Smiths in China.

Now the difference between the really educated person and the half educated one is this.
The really educated person wants to read books which will tell him the real truth about a thing, whereas the half educated person prefers to read books which will tell him what he wants the thing to be, what his vanity prompts him to wish that the thing should be.
John Smith in China wants very much to be a superior person to the Chinaman and the Rev. Arthur Smith writes a book to prove conclusively that he, John Smith, is a very much superior person to the Chinaman.
Therefore\footnote{There-fore in the original book}, the Rev. Arthur Smith is a person very dear to John Smith, and the ``Chinese Characteristics'' become a Bible to John Smith.

But Mr.~W.~Stead\marginpar{\scriptsize William Thomas Stead: (1849-1912)\\ British journalist} says, ``It is John Smith and his neighbours who now rule the British Empire.''
Consequently I have lately taken the trouble to read the books which furnish John Smith with his ideas on China and the Chinese.


The Autocrat at the Breakfast Table\marginpar{\scriptsize The Autocrat of the Breakfast-Table by Oliver Wendell Holmes, Sr.} classified minds under the heads of arithmetical and algebraical intellects.
``All economical and practical wisdom,'' he observes,\footnote{\emph{, ``is} is missing in the original book} ``is an extension or variation of the arithmetical formula 2 plus 2 equal 4. 
Every philosophical proposition has the more general character of the expression $a$ plus $b$ equal $c$.''
Now the whole family of John Smith belong decidedly to the category of minds which the Autocrat calls arithmetical intellects.
John Smith's father, John Smith senr, alias John Bull, made his fortune with the simple formula 2 plus 2 equal 4.
John Bull came to China to sell his Manchester goods and to make money and he got on very well with John Chinaman because both he and John Chinaman understood and agreed perfectly upon the formula 2 plus 2 equal 4.
But John Smith Junr, who now rules the British Empire, comes out to China with his head filled with $a$ plus $b$ equal $c$ which he does not understand --- and not content to sell his Manchester goods, wants to civilise the Chinese or, as he expresses it, to ``spread Anglo-Saxon ideals.''
The result is that John Smith gets on very badly with John Chinaman, and, what is still worse, under the civilising influence of John Smith's $a$ plus $b$ equal $c$ Anglo-Saxon ideals, John Chinaman, instead of being a good, honest, steady customer for Manchester goods neglects his business, goes to Chang Su-ho's Gardens to celebrate the Constitution, in fact becomes a mad, raving reformer.

I have lately, by the help of Mr.~Putnam Weale's\marginpar{\scriptsize Putnam Weale is the pen name of Bertram Lenox Simpson (1877-1930)} ``Reshaping of the Far\footnote{``Far in the original book} East'' and other books, tried to compile a Catechism of Anglo-Saxon\footnote{Anglo Saxon in the original book} Ideals for the use of Chinese students.
The result, so far, is something like this: ---\footnote{without --- in the original book}
\begin{enumerate}
    \item --- What is the chief end of man? \\ The chief end of man is to glorify the British Empire.
    \item --- Do you believe in God? \\ Yes, when I\footnote{1 in the original book} go to Church.
    \item --- What do you believe in when you are\footnote{ere in the original book} not in Church? \\ I believe in interests --- in what will pay.
    \item --- What is justification by faith? \\ To believe in everyone for himself.
    \item --- What is justification by works? \\ Put money in your pocket.
    \item --- What is Heaven? \\ Heaven means to be able to live in Bubbling Well Road\footnote{Roa in the original book}\marginpar{\scriptsize Bubbling Well Road: The most fashionable quarter in Shanghai} and drive in victorias.
    \item --- What is Hell? \\ Hell means to be unsuccessful.
    \item --- What is a state of human perfectibility? \\ Sir Robert Hart's Custom Service in China.
    \item --- What is blasphemy? \\ To say that Sir Robert Hart is not a great man of genius.
    \item --- What is the most heinous sin? \\ To obstruct British trade.
    \item --- For what purpose did God create the four hundred million Chinese? \\ For the British to trade upon.
    \item --- What form of prayer do you use when you pray? \\ We thank Thee, O Lord, that we are not as the wicked Russians and brutal Germans are, who want to partition China.
    \item --- Who is the great Apostle of the Anglo-Saxon Ideals in China. \\ Dr.~Morrison, the \emph{Times} Correspondent in Peking.
\end{enumerate}

It may be a libel to say that the above is a true statement of Anglo-Saxon ideals, but any one who will take the trouble to read Mr.~Putnam Weale's book will not deny that the above is a fair representation of the Anglo-Saxon ideals of Mr.~Putnam Weale and John Smith who reads Mr.~Putnam Weale's books.

The most curious thing about the matter is that the civilising influence of John Smith's\footnote{Smith, s in the original book} Anglo-Saxon ideals is really taking effect in China.
Under this influence John Chinaman too is now wanting to glorify the Chinese Empire.
The old Chinese literati with his eight-legged essays was a harmless humbug.
But foreigners will find to their cost that the new Chinese literati who under the influence of John Smith's Anglo-Saxon ideals is clamouring for a constitution, is likely to become an intolerable and dangerous nuisance.
In the end I fear John Bull Senior will not only find his Manchester goods trade ruined, but he will even be put to the expense of sending out a General Gordon or Lord Kitchener to shoot his poor old friend John Chinaman who has become \emph{non compes mentis} under the civilising influence of John Smith's Anglo-Saxon ideals.
But that is neither here nor there.

What I want to say here in plain, sober English is this.
It is a wonder to me that the Englishman who comes out to China with his head filled with all the arrant nonsense written in books about the Chinese, can get along at all with the Chinese with whom he has to deal.
Take this specimen, for instance, from a big volume, entitled ``The Far East: its history and its questions,'' by Alexis Krausse.

``The crux of the whole question affecting the Powers of the Western nations in the Far East lies in the appreciation of the true inwardness of the Oriental mind.
An Oriental not only sees things from a different standpoint to (!) the Occidental, but his whole train of thought and mode of reasoning are at variance.
The very sense of perception implanted in the Asiatic varies from that with which we are endowed!''

After reading the last sentence an Englishman in China, when he wants a piece of \emph{white} paper, if he follows the ungrammatical Mr.~Krausse's advice, would have to say to his boy: --- ``Boy, bring me a piece of \emph{black} paper.''
It is, I think, to the credit of practical men among foreigners in China that they can put away all this nonsense about the true inwardness of the Oriental mind when they come to deal practically with the Chinese.
In fact I believe that those foreigners get on best with the Chinese and are the most successful men in China who stick to 2 plus 2 equal 4, and leave the $a$ plus $b$ equal $c$ theories of Oriental inwardness and Anglo-Saxon ideals to John Smith and Mr.~Krausse.
Indeed when one remembers that in those old days, before the Rev. Arthur Smith wrote his ``Chinese Characteristics,'' the relations between the heads or taipans\marginpar{\scriptsize taipans: 旧中国洋行老板} of great British firms such as Jardine, Matheson and their Chinese compradores\marginpar{\scriptsize compradores: Chinese employed by foreign firms in China to agents between them and Chinese merchants.} were always those of mutual affection, passing on to one or more generations;
when one remembers this, one is inclined to ask what good, after all, has clever John Smith with his $a$ plus $b$ equal $c$ theories of Oriental inwardness and Anglo-Saxon ideals done, either to Chinese or foreigners?

Is there then no truth in Kipling's\marginpar{\scriptsize Rudyard Kipling (1865-1936) \\ British author and poet} famous dictum that East is East and West is West?
Of course there is.
When you deal with 2 plus 2 equal 4, there is little or no difference.
It is only when you come to problems as $a$ plus $b$ equal $c$ that there is a great deal of difference between East and West.
But to be able to solve the equation $a$ plus $b$ equal $c$ between East and West, one must have real aptitude for higher mathematics.
The misfortune of the world today is that the solution of the equation $a$ plus $b$ equal $c$ in Far Eastern problems, is in the hands of John Smith who not only rules the British Empire, but is an ally of the Japanese nation, --- John Smith who does not understand the elements even of algebraical problems.
The solution of the equation $a$ plus $b$ equal $c$ between East and West is a very complex and difficult problem.
For in it there are many unknown quantities, not only such as the East of Confucius and the East of Mr.~Kang Yu-wei\marginpar{\scriptsize Mr.~Kang Yu-wei: \\ 康有为} and the Viceroy Tuan Fang\marginpar{\scriptsize Viceroy Tuan Fang: \\ 总督}, but also the West of Shakespeare and Goethe and the West of John Smith.
Indeed when you have solved your $a$ plus $b$ equal $c$ equation properly, you will find that there is very little difference between the East of Confucius and the West of Shakespeare and Goethe, but you will find a great deal of difference between even the West of Dr.~Legge\marginpar{\scriptsize James Legge (1815-1897) \\ Scottish sinologist} the scholar, and the West of the Rev. Arthur Smith.
Let me give a concrete illustration of what I mean.

The Rev. Arthur Smith, speaking of Chinese histories, says: ---\footnote{without --- in the original book}

``Chinese histories are antediluvian\marginpar{\scriptsize 上古的}, not merely in their attempts to go back to the ragged edge of zero of time for a point of departure, but in the interminable length of the sluggish and turbid current which carries on its bosom not only the mighty vegetation of past ages, but wood, hay and stubble past all reckoning. None but a relatively timeless race could either compose or read such histories: none but the Chinese memory could store them away in its capacious abdomen!''

Now let us hear Dr.~Legge on the same subject.
Dr.~Legge\footnote{lcgge in the original book}, speaking of the 23 standard dynastic histories of China, says: ---\footnote{without --- in the original book}

``No nation has a history so thoroughly digested; and on the whole it is trustworthy.''

Speaking of another great Chinese literary collection, Dr.~Legge says: --- 

``The work was not published, as I once supposed by Imperial authority, but under the superintendence and at the expense (aided by other officers) of Yuen Yun, Governor-General of Kwangtung\marginpar{\scriptsize Kwangtung: 广东} and Kwangse\marginpar{\scriptsize Kwangse: 广西}, in the 9th year of the last reign, of Kien-lung 1820.
The publication of so extensive a work shows a \emph{public spirit and zeal for literature} among the high officials of China which should keep foreigners from thinking meanly\footnote{meanly of in the original book} them.''

The above then is what I mean when I say that there is a great deal of difference not only between the East and West but also between the West of Dr.~Legge, the scholar who can appreciate\footnote{appreaate in the original book} and admire zeal for literature, and the West of the Rev. Arthur Smith who is the beloved of John Smith in China.
