\chapter{A Great Sinologue}
\begin{center}
    \scriptsize Don't forget to be a gentleman of sense, while you try to be a great scholar; \\ Don't become a fool, while you try to be a great scholar. \\
    \hfill Confucius Sayings, Ch: VI. $_{\text{II}.}$
\end{center}

I have lately been reading Dr.~Giles'\marginpar{\scriptsize Herbert Allen Giles: (1845-1935) a British diplomat and sinologist} ``Adversaria Sinica,''\footnote{'' is missing in the original book} and in reading them, was reminded of a saying of another British Consul Mr.~Hopkins that ``when foreign residents in China speak of a man as a sinologue, they generally think of him as a fool.''

Dr.~Giles\footnote{Giles' in the original book} has the reputation of being a great Chinese scholar.
Considering the quantity of work he has done, that reputation is not undeserved.
But I think it is now time that an attempt should be made to accurately estimate the quality and real value of Dr.~Giles'\footnote{Giles in the original book} work.

In one respect Dr.~Giles has the advantage over all sinologues past and present, --- he possesses the literary gift: he can write good idiomatic English.
But on the other hand Dr.~Giles utterly lacks the philosophical insight and sometimes even common sense.
He can translate Chinese sentences, but he cannot interpret and understand Chinese thought.
In this respect, Dr.~Giles has the same characteristics as the Chinese literati.
Confucius says, ``When men's education or book learning get the better of their natural qualities, they become \emph{literati}'' (Chap. VI. 16.)

To the Chinese literati, books and literature are merely materials for writing books and so they write books upon books.
They live, move and have their being in a world of books, having nothing to do with the world of real human life.
It never occurs to the literati that books and literature are only means to an end.
The study of books and literature to the true scholar is but the means to enable him to interpret, to criticise, to understand human life.

Matthew Arnold says, ``It is through the apprehension either of all literature, --- the entire history of the human spirit, --- or of a single great literary work as a connected whole, that the power of literature makes itself felt.''
But in all that Dr.~Giles has written, there is not a single sentence which betrays the fact that Dr.~Giles has conceived or even tried to conceive the Chinese literature as a connected whole.

It is this want of philosophical insight in Dr.~Giles which makes him so helpless in the arrangement of his materials in his books.
Take for instance his great dictionary.
It is in no sense a dictionary at all.
It is merely a collection of Chinese phrases and sentences, translated by Dr.~Giles without any attempt at selection, arrangement, order or method.
As a dictionary for the purposes of the scholar, Dr.~Giles' dictionary is decidedly of less value than even the old dictionary of Dr.~Williams.

Dr.~Giles' Chinese biographical dictionary, it must be admitted, is a work of immense labour.
But here again Dr.~Giles shows an utter lack of the most ordinary judgment.
In such a work, one would expect to find notices only of really notable men.
\begin{center}
    Hic manus ob patriam pugnando vulnera passi, \\
    Quique sacerdotes casti, dum vita manebat, \\
    Quique pii vates et Phoebo\footnote{Phxwbo in the original book} digna locuti\footnote{loctlti in the original book}, \\
    Inventas aut qui vitam\footnote{intam in the original book} excoluere per artes, \\
    Quique sui memores aliquos fecere merendo.
\end{center}

But side by side with sages and heroes of antiquity, with mythical and mythological personages, we find General Tcheng Ki-tong\marginpar{\scriptsize Tcheng Ki-tong: 陈季同}, Mr.~Ku Hung-ming, Viceroy Chang Chi-tung\marginpar{\scriptsize Chang Chi-tung: 张之洞} and Captain Lew Buah, --- the last whose sole title to distinction is that he used often to treat his foreign friends with unlimited quantities of champagne!

Lastly these ``Adversaria,'' --- Dr.~Giles\footnote{Giles' in the original book} latest publication --- will not, I am afraid, enhance Dr.~Giles\footnote{Giles' in the original book} reputation as a scholar of sense and judgment.
The subjects chosen, for the most part, have no earthly practical or human interest.
It would really seem that Dr.~Giles has taken the trouble to write these books not with any intention to tell the world anything about the Chinese and their literature but to show what a learned Chinese scholar Dr.~Giles is and how much better he understands Chinese than anybody else.
Moreover, Dr.~Giles, here as elsewhere, shows a harsh and pugnacious dogmatism which is as unphilosophical, as unbecoming a scholar as it is unpleasing.
It is these characteristics of sinologues like Dr.~Giles which have made, as Mr.~Hopkins says, the very name of sinologue and Chinese scholarship a byword and scorn among practical foreign residents in the Far East.

I shall here select two articles from Dr.~Giles\footnote{Giles' in the original book} latest publication and will try to show that if hitherto writings of foreign scholars on the subjects of Chinese learning and Chinese literature have been without human or practical interest, the fault is not in Chinese learning and Chinese literature.

The first article is entitled ``What is filial piety.''
The point in the article turns upon the meaning of two Chinese characters.
A disciple asked what is filial piety.
Confucius said: \emph{se nan} 色难 (lit, colour difficult).

Dr.~Giles says, ``The question is, and has been for twenty centuries past, what do these two characters mean?''
After citing and dismissing all the interpretations and translations of native and foreign scholars alike, Dr.~Giles of course finds out the true meaning.
In order to show Dr.~Giles harsh and unscholarly dogmatic manner, I shall here quote Dr.~Giles' words with which he announces his discovery. Dr.~Giles says: ---

``It may seem presumptuous after the above exordium to declare that the meaning lies \`a la Bill Stumps (!) upon the surface, and all you have to do, as the poet says, is to
\begin{center}
    Stoop, and there it is; \\
    Seek it not right nor left!''
\end{center}

When Tzu-hsia asked Confucius, `What is filial piety?' the latter replied simply,

``\thinspace`\emph{se} (色) to define it, \emph{nan} (难) is difficult,' a most intelligible and appropriate answer.''

I shall not here enter into the niceties of Chinese grammar to\footnote{lo in the original book} show that Dr.~Giles is wrong.
I will only say here that if Dr.~Giles is right in supposing that the character \emph{se} (色) is a verb, then in good grammatical Chinese, the sentence would not read \emph{se nan} (色难), but \emph{se chih wei nan} (色之维难) to define it, is difficult.
The impersonal pronoun' \emph{chih} (之) \emph{it}, is here absolutely indispensable, if the character \emph{se} (色) here is used as a verb.

But apart from grammatical niceties, the translation as given by Dr.~Giles of Confucius answer, when taken with the whole context, has no point or sense in it at all.

Tzu hsia asked, what is filial piety?
Confucius said, ``The difficulty is with the \emph{manner}\cite{num25} of doing it. That merely when there is work to be done, the young people should take the trouble of doing it, and when there is wine and food, the old folk are allowed to partake it, --- do you really think that is filial piety?'' (Discourses and Sayings Ch. II. 9.)
Now the whole point in the text above lies in this, --- that importance is laid not upon \emph{what} duties you perform towards your parents, but upon \emph{how} --- in what manner, with what spirit, you perform those duties.

The greatness and true efficacy of Confucius'\footnote{Confucius's in the original book} moral teaching, I wish to say here, lies in this very point which Dr.~Giles fails to see, --- the point namely that in the performance of moral duties, Confucius insisted upon\footnote{upon, in the original book} the importance not of the \emph{what}, but of the \emph{how}.
For herein lies the difference between what is called morality and religion, between mere rules of moral conduct and the vivifying teaching of great and true religious teachers.
Teachers of morality merely tell you what kind of action is moral and what kind of action is immoral.
But true religious teachers do not merely tell you this.
True religious teachers do not merely inculcate the doing of the outward act, but insist upon the importance of the manner, the \emph{inwardness} of the act.
True religious teachers teach that the morality or immorality of our actions does not consist in \emph{what} we do, but in \emph{how} we do it.

This is what Matthew Arnold calls Christ's method in his teaching.
When the poor widow gave her mite, it was not \emph{what} she gave that Christ called the attention of his hearers to, but \emph{how} she give\footnote{gave in the original book} it.
The moralists said, ``Thou shalt not commit adultery.''
But Christ said, ``I say unto you that whosoever looketh\footnote{looks in the original book} on a woman to lust after her hath already committed adultery.''

In the same way the moralists in Confucius' time said: Children must cut firewood and carry water for their parents and yield to them the best of the food and wine in the house: that is filial piety.
But Confucius said, ``No; that is \emph{not} filial piety.''
True filial piety does not consist in the mere outward performance of these services to our parents.
True filial piety consists in \emph{how}, in what manner, with what spirit we perform these services.
The difficulty, said Confucius, is with the \emph{manner} of doing it.
It is, I will finally say here, by virtue of this method in his teaching, of looking into the inwardness of moral actions that Confucius becomes, not as the Christian missionaries say, a mere moralist and philosopher, but a great and true religious teacher.

As a further illustration of Confucius method, take the present reform movement in China.
The so called progressive mandarins with applause from foreign newspapers are making a great fuss --- even going to Europe and America, --- trying to find out what reforms to adopt in China.
But unfortunately the salvation of China will not depend upon \emph{what} reforms are made by these progressive mandarins, but upon how these reforms are carried out.
It seems a pity that these progressive mandarins, --- instead of going to Europe and America, to study constitution could not be made to stay at home and study Confucius.
For until these mandarins take to heart Confucius' teaching and his method and attend to the \emph{how} instead of the \emph{what} in this matter of reform, nothing but chaos, misery and suffering will come out of the present reform movement in China.

The other article in Dr.~Giles ``Adversaria Sinica'' which I will briefly examine, is entitled --- ``The four classes.''

The Japanese Baron Suyematzu in an interview said that the Japanese divided their people into four classes, --- soldiers, farmers, artisans and warriors.
Upon this Dr.~Giles says. ``It is incorrect to translate \emph{shih} (士) as soldier; that is a later meaning.''
Dr.~Giles further says, ``in its earliest use the word \emph{shih} (士) referred to civilians.''

Now the truth is just on the other side.
In its earliest use, the word \emph{shih} (士) referred to gentlemen who in ancient China, as it is now in Europe, bore arms, --- the \emph{noblesse} of the sword.
Hence the officers and soldiers of an army were spoken of as \emph{shih tsu} (士卒).

The civilian official class in ancient China were called \emph{shi} (史) --- clericus.
When the feudal system in China was abolished (2nd cent. B.C.) and fighting ceased to be the only profession of gentlemen, this civilian official class rose into prominence, became lawyers and constituted the \emph{noblesse} of the robe as distinguished from the \emph{shih} (士) the \emph{noblesse} of the sword.

H.E. the Viceroy Chang of Wuchang once asked me why the foreign consuls who were civil functionaries, when in full dress, wore swords.
In reply I said that it was because they were \emph{shih} (士) which in ancient China meant not a civilian scholar, but a gentleman who bore arms and served in the army.
H.E. agreed and the next day gave orders that all the pupils in the schools in Wuchang should wear military uniform.

This question therefore which Dr.~Giles has raised whether the Chinese word \emph{shih} (士) means a civilian or a military man has a great practical interest.
For the question whether China in the future will be independent or come under a foreign yoke will depend upon whether she will ever have an efficient army and that question again will depend upon whether the educated and governing class in China will ever regain the true ancient meaning and conception of the word \emph{shih} (士) not as civilian scholar, but as a gentleman who bears arms and is able to defend his country against aggression. 


